\section{Uitbeidingen op TikZ}

\begin{frame}
  \frametitle{Een kort overzichtje}

  Zie \url{http://ctan.org/search?phrase=tikz}: 90+ resultaten

  \begin{enumerate}
    \item \texttt{tikz-qtree}
    \item \texttt{tqft}
    \item \texttt{tikz-cd}
  \end{enumerate}
\end{frame}

\begin{frame}
  \frametitle{\texttt{tikz-qtree}}
  
  Automatisch bomen tekenen:
  \begin{columns}
    \begin{column}{.6\textwidth}
      \inputminted[fontsize = \scriptsize]{latex}{tikz/tikz-qtree.tikz}
    \end{column}
    \begin{column}{.4\textwidth}
      \begin{tikzpicture}[scale = .8]
  \Tree [.S [.NP [.Det the ] [.N cat ] ]
            [.VP [.V sat ]
                 [.PP [.P on ]
                      [.NP [.Det the ]
                      [.N mat ] ] ] ] ]
\end{tikzpicture}

    \end{column}
  \end{columns}
\end{frame}

\begin{frame}
  \frametitle{\texttt{tqft}}
  
  Om diagrammen in topologische quantumveldentheorie te tekenen:
  \begin{columns}
    \begin{column}{.6\textwidth}
      \inputminted[fontsize = \scriptsize]{latex}{tikz/tqft.tikz}
    \end{column}
    \begin{column}{.4\textwidth}
      \begin{tikzpicture}[scale = .8]
  \node[draw, tqft/pair of pants]
    (a) {};
  \node[draw, tqft/cylinder to next,
        anchor = incoming boundary 1]
    (c) at (a.outgoing boundary 1) {};
  \node[draw, tqft/reverse pair of pants,
        anchor = incoming boundary 1]
    at (a.outgoing boundary 2) (b) {};
\end{tikzpicture}

    \end{column}
  \end{columns}
\end{frame}

\section{Commutatieve diagrammen}

\begin{frame}[fragile]
  \frametitle{Commutatieve diagrammen}

  \small
  We gebruiken \mintinline{latex}|\usepackage{tikz-cd}|.
  \begin{columns}
    \begin{column}{.67\textwidth}
      \inputminted[fontsize = \scriptsize]{latex}{tikz/diagrams/1.tikz}
    \end{column}
    \begin{column}{.33\textwidth}
      \begin{equation*}
  \begin{tikzcd}
    A \arrow{rd} \arrow{r}{\phi} & B \\
                                 & C
  \end{tikzcd}
\end{equation*}

    \end{column}
  \end{columns}
  \small
  \begin{enumerate}
    \item\pause de \emph{eerste parameter} van \mintinline{latex}|\arrow| is altijd de richting: een combinatie van \texttt{u}, \texttt{d}, \texttt{r}, \texttt{l}, de \emph{tweede} een (optioneel) label
    \item\pause vertices zoals we tabellen schrijven 
  \end{enumerate}
  \begin{alertblock}{Opgepast}
    \dbend\quad Pijlen kunnen enkel naar bestaande vertices.
  \end{alertblock}
\end{frame}

