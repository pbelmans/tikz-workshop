\begin{frame}
  \frametitle{Waarom figuren maken in \LaTeX?}

  \begin{itemize}
    \item[+] aanpasbaarheid
      \begin{enumerate}
        \item je document opnieuw builden maakt de afbeelding opnieuw: wijzigingen aanbrengen is dus even gemakkelijk als \LaTeX-code wijzigen
        \item je hebt geen externe software nodig (!)
      \end{enumerate}
    \item[+] consistentie
      \begin{enumerate}
        \item lettertypes, kleuren, tekstgroottes, \ldots zijn overal hetzelfde
      \end{enumerate}
  \end{itemize}
\end{frame}

\begin{frame}
  \frametitle{Waarom figuren maken in \TikZ?}
  \begin{exampleblock}{}
    \mintinline{latex}|Making Greek letters is as easy as $\pi$.|

    Making Greek letters is as easy as $\pi$. 
    \vskip5mm
    \hspace*\fill{---Leslie Lamport, \LaTeX: A Document Preparation System}
  \end{exampleblock}
  \pause
  \begin{exampleblock}{}
    \mintinline{latex}|Drawing an orange circle is as easy as|
    \mintinline{latex}|\tikz \fill[orange] (1ex,1ex) circle (1ex);.|

    Drawing an orange circle is as easy as \tikz \fill[orange] (1ex,1ex) circle (1ex);.
    \vskip5mm
    \hspace*\fill{---Pieter Belmans}
  \end{exampleblock}
\end{frame}

\begin{frame}
  \frametitle{Wat is \TikZ?}

  \TikZ is een recursief acroniem voor
  \begin{center}
    \TikZ ist \emph{kein} Zeichenprogramm
  \end{center}
  \pause
  Dit betekent vooral dat \TikZ niet bedoeld is om in te tekenen zoals in Paint, maar om ``vector graphics'' te produceren: we ``beschrijven'' de tekening.
  \pause
  Eigenlijk is \TikZ een frontend voor PGF, Portable Graphics Format, soms wordt de combinatie PGF/\TikZ genoemd.
  \pause
  \begin{block}{Versies}
    Enkele maanden geleden is versie 3 uitgekomen, maar deze workshop is nog geschreven voor 2.10 (de versie die ge\"installeerd is).
  \end{block}
\end{frame}

\begin{frame}
  \frametitle{Wat gaan we vandaag vooral \emph{niet} doen?}

  Dit is \emph{geen} grondige \TikZ-cursus:
  \begin{enumerate}
    \item ik ben verre van een expert
    \item de tijd is beperkt
    \item \TikZ is \emph{gigantisch}
    \item\pause ik ben verre van een expert
  \end{enumerate}
\end{frame}

\begin{frame}
  \frametitle{Wat gaan we vandaag dan \emph{wel} doen?}

  \begin{enumerate}
    \item korte introductie tot de syntax
    \item waar kunnen we informatie vinden?
    \item wat is er allemaal mogelijk?
  \end{enumerate}
  Ik wil dus vooral laten zien \emph{wat} er mogelijk is, niet \emph{hoe} je het moet doen. Daarvoor dienen handleidingen en de vele beschikbare voorbeelden.
\end{frame}

