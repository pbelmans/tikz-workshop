\section{PGF/TikZ}

\begin{frame}
  \frametitle{Uitgewerkt voorbeeld: een gelijkzijdige driehoek}

  In de \emph{Elementen van Euclides} (3e eeuw voor Christus) worden vele constructies in vlakke meetkunde beschreven. We gaan nu een voorbeeld uit de Ti\textit{k}Z manual bespreken:
  \begin{quote}
    Hoe construeren we een gelijkzijdige driehoek op een gegeven lijnstuk?
  \end{quote}

  \pause
  \centering
  \begin{tikzpicture}[thick,help lines/.style={thin,draw=black!50}]
  \def\A{\textcolor{input}{$A$}}
  \def\B{\textcolor{input}{$B$}}
  \def\C{\textcolor{output}{$C$}}
  \def\D{$D$}
  \def\E{$E$}

  \colorlet{input}{blue!80!black}
  \colorlet{triangle}{orange!80!white}
  \colorlet{output}{red!70!black}

  \coordinate [label=left:\A]  (A) at ($ (0,0) + .1*(rand,rand) $);
  \coordinate [label=right:\B] (B) at ($ (1.25,0.25) + .1*(rand,rand) $);

  \draw [input] (A) -- (B);

  \node [name path=D,help lines,draw,label=left:\D]
    (D) at (A) [circle through=(B)] {};
  \node [name path=E,help lines,draw,label=right:\E]
    (E) at (B) [circle through=(A)] {};

  \path [name intersections={of=D and E,by={[label=above:\C]C}}];

  \draw [output] (A) -- (C) -- (B);

  \foreach \point in {A,B,C}
    \fill [black,opacity=.5] (\point) circle (2pt);

  \begin{pgfonlayer}{background}
    \fill[triangle!80] (A) -- (C) -- (B) -- cycle;
  \end{pgfonlayer}
\end{tikzpicture}

\end{frame}

\begin{frame}
  \frametitle{Stappenplan}

  \begin{columns}
    \begin{column}{.5\textwidth}
      \begin{enumerate}
        \item de rechte $AB$
        \item de cirkels rond $A$ en $B$
        \item het snijpunt
        \item de driehoek
      \end{enumerate}
    \end{column}
    \begin{column}{.5\textwidth}
      \centering
      \begin{tikzpicture}[thick,help lines/.style={thin,draw=black!50}]
  \def\A{\textcolor{input}{$A$}}
  \def\B{\textcolor{input}{$B$}}
  \def\C{\textcolor{output}{$C$}}
  \def\D{$D$}
  \def\E{$E$}

  \colorlet{input}{blue!80!black}
  \colorlet{triangle}{orange!80!white}
  \colorlet{output}{red!70!black}

  \coordinate [label=left:\A]  (A) at ($ (0,0) + .1*(rand,rand) $);
  \coordinate [label=right:\B] (B) at ($ (1.25,0.25) + .1*(rand,rand) $);

  \draw [input] (A) -- (B);

  \node [name path=D,help lines,draw,label=left:\D]
    (D) at (A) [circle through=(B)] {};
  \node [name path=E,help lines,draw,label=right:\E]
    (E) at (B) [circle through=(A)] {};

  \path [name intersections={of=D and E,by={[label=above:\C]C}}];

  \draw [output] (A) -- (C) -- (B);

  \foreach \point in {A,B,C}
    \fill [black,opacity=.5] (\point) circle (2pt);

  \begin{pgfonlayer}{background}
    \fill[triangle!80] (A) -- (C) -- (B) -- cycle;
  \end{pgfonlayer}
\end{tikzpicture}

    \end{column}
  \end{columns}
\end{frame}

\begin{frame}
  \frametitle{Euclides: de configuratie}
  
  \inputminted[fontsize = \scriptsize]{latex}{tikz/triangle/configuration.tikz}

  \TikZ bevat vele libraries die bepaalde zaken gemakkelijker maken. In dit voorbeeld zullen we er vier gebruiken.
\end{frame}

\begin{frame}[fragile]
  \frametitle{Euclides: de rechte $AB$ (1)}

  \begin{columns}
    \begin{column}{.67\textwidth}
      \inputminted[fontsize = \scriptsize]{latex}{tikz/triangle/1a.tikz}
    \end{column}
    \begin{column}{.33\textwidth}
      \begin{tikzpicture}
  \coordinate (A) at (0,0);
  \coordinate (B) at (1.25,0.25);
  \draw[blue] (A) -- (B);
\end{tikzpicture}

    \end{column}
  \end{columns}

  \small
  \begin{enumerate}
    \item\pause \mintinline{latex}|\draw| is het commando om iets te tekenen
    \item\pause \mintinline{latex}|--| is een gewone lijn tussen twee punten
    \item\pause we konden ook \mintinline{latex}|\draw[blue] (0,0) -- (1.25,0.25);| gebruiken, maar nu kunnen we de co\"ordinaten herbruiken
  \end{enumerate}
  \pause
  \begin{block}{Volgende stap}
    Hoe voegen we namen van punten toe?
  \end{block}
\end{frame}

\begin{frame}
  \frametitle{Euclides: de rechte $AB$ (2)}

  \begin{columns}
    \begin{column}{.67\textwidth}
      \inputminted[fontsize = \scriptsize]{latex}{tikz/triangle/1b.tikz}
    \end{column}
    \begin{column}{.33\textwidth}
      \begin{tikzpicture}
  \coordinate[label=left:\textcolor{blue}{$A$}]
    (A) at (0,0);
  \coordinate[label=right:\textcolor{blue}{$B$}]
    (B) at (1.25,0.25);

  \draw[blue] (A) -- (B);
\end{tikzpicture}

    \end{column}
  \end{columns}

  \small
  \begin{enumerate}
    \item\pause we gebruiken puntkomma's voor het einde van een commando
    \item\pause we mogen nieuwe regels beginnen als dat beter leest (of als de slides te smal zijn)
  \end{enumerate}
  \pause
  \begin{block}{Volgende stap}
    Hoe kunnen we $A$ en $B$ een willekeurige positie laten aannemen?
  \end{block}
\end{frame}

\begin{frame}
  \frametitle{Euclides: de rechte $AB$ (3)}

  \begin{columns}
    \begin{column}{.67\textwidth}
      \inputminted[fontsize = \scriptsize]{latex}{tikz/triangle/1c.tikz}
    \end{column}
    \begin{column}{.33\textwidth}
      \begin{tikzpicture}
  \coordinate [label=left:\textcolor{blue}{$A$}]
    (A) at (0+0.1*rand,0+0.1*rand);
  \coordinate [label=right:\textcolor{blue}{$B$}]
    (B) at (1.25+0.1*rand,0.25+0.1*rand);

  \draw[blue] (A) -- (B);
\end{tikzpicture}

    \end{column}
  \end{columns}

  \small
  \begin{enumerate}
    \item\pause we kunnen dus berekeningen doen bij het bepalen van co\"ordinaten
    \item\pause \texttt{rand} is een functie die een getal tussen $-1$ en $1$ teruggeeft
  \end{enumerate}
  \pause
  \begin{block}{Volgende stap}
    We hebben basispunten en een perturbatie, maar de syntax is een beetje onduidelijk.
  \end{block}
\end{frame}

\begin{frame}
  \frametitle{Euclides: de rechte $AB$ (4)}

  \begin{columns}
    \begin{column}{.67\textwidth}
      \inputminted[fontsize = \scriptsize]{latex}{tikz/triangle/1d.tikz}
    \end{column}
    \begin{column}{.33\textwidth}
      \begin{tikzpicture}
  \coordinate [label=left:\textcolor{blue}{$A$}]
    (A) at ($ (0,0) + .1*(rand,rand) $);
  \coordinate [label=right:\textcolor{blue}{$B$}]
    (B) at ($ (1.25,0.25) + .1*(rand,rand) $);

  \draw[blue] (A) -- (B);
\end{tikzpicture}

    \end{column}
  \end{columns}

  \small
  \begin{enumerate}
    \item\pause de library \texttt{calc} laat ons toe om heel flexibel met co\"ordinaten te rekenen: we gebruiken dollartekens om dit aan te duiden
    \item\pause meestal is \texttt{calc} zelfs niet nodig (!)
  \end{enumerate}
  \pause
  \begin{block}{Volgende stap}
    Nu de cirkels.
  \end{block}
\end{frame}

\begin{frame}[fragile]
  \frametitle{Euclides: de cirkels rond $A$ en $B$ (1)}

  \begin{columns}
    \begin{column}{.67\textwidth}
      \inputminted[fontsize = \scriptsize]{latex}{tikz/triangle/2a.tikz}
    \end{column}
    \begin{column}{.33\textwidth}
      \begin{tikzpicture}
  \coordinate [label=left:$A$]  (A) at (0,0);
  \coordinate [label=right:$B$] (B) at (1.25,0.25);
  \draw (A) -- (B);

  \draw (A) let
              \p1 = ($ (B) - (A) $)
            in
              circle ({veclen(\x1,\y1)});
\end{tikzpicture}

    \end{column}
  \end{columns}

  \small
  \begin{enumerate}
    \item\pause een cirkel tekenen
      \mintinline{latex}|\draw (0,0) circle (1);|
    \item\pause we berekenen de co\"ordinaten voor $B-A$ (\texttt{let \ldots\ in} is met variabelen werken)
  \end{enumerate}
  \pause
  \begin{block}{Volgende stap}
    Nu nog de tweede cirkel.
  \end{block}
\end{frame}

\begin{frame}[fragile]
  \frametitle{Euclides: de cirkels rond $A$ en $B$ (2)}

  \begin{columns}
    \begin{column}{.67\textwidth}
      \inputminted[fontsize = \scriptsize]{latex}{tikz/triangle/2b.tikz}
    \end{column}
    \begin{column}{.33\textwidth}
      \begin{tikzpicture}
  \coordinate[label=left:$A$]  (A) at (0,0);
  \coordinate[label=right:$B$] (B) at (1.25,0.25);
  \draw (A) -- (B);

  \draw let \p1 = ($ (B) - (A) $),
            \n2 = {veclen(\x1,\y1)}
        in
          (A) circle (\n2)
          (B) circle (\n2);
\end{tikzpicture}

    \end{column}
  \end{columns}

  \small
  \begin{enumerate}
    \item\pause \mintinline{latex}|\p1| zijn co\"ordinaten, \mintinline{latex}|\n1| is een getal
    \item\pause \'e\'en \mintinline{latex}|\draw| kan meerdere zaken tekenen
  \end{enumerate}
  \pause
  \begin{block}{Volgende stap}
    We kunnen echter ook meteen cirkels door punten tekenen.
  \end{block}
\end{frame}

\begin{frame}[fragile]
  \frametitle{Euclides: de cirkels rond $A$ en $B$ (3)}

  \begin{columns}
    \begin{column}{.67\textwidth}
      \inputminted[fontsize = \scriptsize]{latex}{tikz/triangle/2c.tikz}
    \end{column}
    \begin{column}{.33\textwidth}
      \begin{tikzpicture}
  \coordinate[label=left:$A$]  (A) at (0,0);
  \coordinate[label=right:$B$] (B) at (1.25,0.25);
  \draw (A) -- (B);

  \node[draw,circle through=(B),label=left:$D$]
    at (A) {};
  \node[draw,circle through=(A),label=right:$E$]
    at (B) {};
\end{tikzpicture}

    \end{column}
  \end{columns}

  \small
  \begin{enumerate}
    \item \mintinline{latex}|\node| is een combinatie van \mintinline{latex}|\coordinate| en \mintinline{latex}|\draw|
    \item de lege accolades zouden gebruikt kunnen worden voor tekst
  \end{enumerate}
  \pause
  \begin{block}{Volgende stap}
    Het snijpunt bepalen.
  \end{block}
\end{frame}

\begin{frame}
  \frametitle{Euclides: het snijpunt (1)}

  \begin{columns}
    \begin{column}{.67\textwidth}
      \inputminted[fontsize = \scriptsize]{latex}{tikz/triangle/3a.tikz}
    \end{column}
    \begin{column}{.33\textwidth}
      \begin{tikzpicture}
  \coordinate[label=left:$A$]  (A) at (0,0);
  \coordinate[label=right:$B$] (B) at (1.25,0.25);
  \draw (A) -- (B);

  \node (D) [name path=D,draw,circle through=(B),
             label=left:$D$]  at (A) {};
  \node (E) [name path=E,draw,circle through=(A),
             label=right:$E$] at (B) {};

  \path[name intersections={of=D and E}];

  \coordinate[label=above:$C$] (C) at (intersection-1);

  \draw[red] (A) -- (C);
  \draw[red] (B) -- (C);
\end{tikzpicture}

    \end{column}
  \end{columns}
\end{frame}

\begin{frame}[fragile]
  \frametitle{Euclides: het snijpunt (2)}

  \begin{enumerate} 
    \item we berekenen een \mintinline{latex}|\path| (= een reeks lijnsegmenten), dus in dit geval het (unieke) lijnstuk tussen de doorsnedes
    \item\pause we hebben de variabelen \texttt{intersection-1} en \texttt{intersection-2} die aangemaakt worden door de \texttt{intersections} library
    \item\pause we introduceren de naam \texttt{C} en tekenen twee lijnstukken
  \end{enumerate}
\end{frame}

\begin{frame}
  \frametitle{Euclides: de driehoek}

  \begin{columns}
    \begin{column}{.67\textwidth}
      \inputminted[fontsize = \scriptsize]{latex}{tikz/triangle/4a.tikz}
    \end{column}
    \begin{column}{.33\textwidth}
      \begin{tikzpicture}
  \coordinate [label=left:$A$]  (A) at (0,0);
  \coordinate [label=right:$B$] (B) at (1.25,0.25);
  \draw (A) -- (B);

  \node (D) [name path=D,draw,circle through=(B),
             label=left:$D$]  at (A) {};
  \node (E) [name path=E,draw,circle through=(A),
             label=right:$E$] at (B) {};

  \path [name intersections={of=D and E}];

  \coordinate [label=above:$C$] (C) at (intersection-1);

  \draw [red] (A) -- (C);
  \draw [red] (B) -- (C);

  \draw [orange, fill] (A) -- (B) -- (C) -- cycle;
\end{tikzpicture}

    \end{column}
  \end{columns}
\end{frame}

\begin{frame}[fragile]
  \frametitle{Euclides: perfectioneren}

  \begin{enumerate}
    \item we gebruiken variabelen voor de kleuren:
      \begin{minted}{latex}
\colorlet{input}{blue!80!black}
\colorlet{triangle}{orange!80!white}
\colorlet{output}{red!70!black}
      \end{minted}
      (en passen de code aan)
    \item\pause we tekenen onze punten:
      \begin{minted}{latex}
\foreach \point in {A,B,C}
  \fill [black,opacity=.5] (\point) circle (2pt);
      \end{minted}
  \end{enumerate}
\end{frame}

\begin{frame}
  \frametitle{Euclides: het resultaat}

  \centering
  \begin{tikzpicture}[scale=1.6,thick,help lines/.style={thin,draw=black!50}]
  \def\A{\textcolor{input}{$A$}}
  \def\B{\textcolor{input}{$B$}}
  \def\C{\textcolor{output}{$C$}}
  \def\D{$D$}
  \def\E{$E$}

  \colorlet{input}{blue!80!black}
  \colorlet{triangle}{orange!80!white}
  \colorlet{output}{red!70!black}

  \coordinate [label=left:\A]  (A) at ($ (0,0) + .1*(rand,rand) $);
  \coordinate [label=right:\B] (B) at ($ (1.25,0.25) + .1*(rand,rand) $);

  \draw [input] (A) -- (B);

  \node [name path=D,help lines,draw,label=left:\D]
    (D) at (A) [circle through=(B)] {};
  \node [name path=E,help lines,draw,label=right:\E]
    (E) at (B) [circle through=(A)] {};

  \path [name intersections={of=D and E,by={[label=above:\C]C}}];

  \draw [output] (A) -- (C) -- (B);

  \foreach \point in {A,B,C}
    \fill [black,opacity=.5] (\point) circle (2pt);

  \begin{pgfonlayer}{background}
    \fill[triangle!80] (A) -- (C) -- (B) -- cycle;
  \end{pgfonlayer}
\end{tikzpicture}


\end{frame}

\begin{frame}
  \frametitle{Herhaling belangrijke commando's}
\end{frame}

